\documentclass[journal,twocolumn,12pt]{IEEEtran}

\usepackage[utf8]{inputenc}
\usepackage{graphicx}
\usepackage{amssymb}
\usepackage{mathtools}
\usepackage{diffcoeff}
\usepackage{amsmath}
\providecommand{\pr}[1]{\ensuremath{\Pr\left(#1\right)}}
\providecommand{\brak}[1]{\ensuremath{\left(#1\right)}}
\newcommand{\Int}{\int\limits}

\title{Assignment 11}
\author{Gollapudi Sasank CS21BTECH11019}

\begin{document}
\maketitle
\section*{Question : }
Show that if the random variables $x,y,$ and $z$ are jointly normal and independent in pairs, they are independent.
\section*{Solution : }
First here we assume $\eta_x = 0 ,\eta_y = 0 ,\eta_z = 0 $ where $\eta_x,\eta_y,\eta_z$ are the mean values of the random variables $x,y,z$ respectively.\\
Given $x,y,z$ are jointly normal and pairwise independent.\\
$x = N(0,\sigma_x) , y = N(0,\sigma_y) , z = N(0,\sigma_z)$\\
\begin{align}
f_x(x) &= \frac{1}{\sigma_x\sqrt{2\pi}} e^{-{(x-\eta_x)}^{2}/2        {\sigma_x}^{2}} \\
\Rightarrow f_x(x) &= \frac{1}{\sigma_x\sqrt{2\pi}} e^{-x^{2}/2{\sigma_x}^{2}} 
\end{align}
Similarly 
\begin{align}
f_y(y) &= \frac{1}{\sigma_y\sqrt{2\pi}} e^{-y^{2}/2{\sigma_y}^{2}} \\
f_z(z) &= \frac{1}{\sigma_z\sqrt{2\pi}} e^{-z^{2}/2{\sigma_z}^{2}}
\end{align}
We know that if the random variables $x_i$ for $i = 1,2,3...n$ are jointly normal 
\begin{align}
f(x_1,x_2,.....,x_n) &= \frac{1}{\sigma_1\sigma_2......\sigma_n \sqrt{{(2\pi)}^n}} e^{{-\frac{1}{2} \brak{\frac{x_1^2}{\sigma_1^2} +..... \frac{x_n^2}{\sigma_n^2} } }} \\
\Rightarrow f(x,y,z) &= \frac{1}{\sigma_x\sigma_y\sigma_z \sqrt{{(2\pi)}^3}} e^{-\frac{1}{2} \brak{ \frac{x^2}{\sigma_x^2} + \frac{y^2}{\sigma_y^2} + \frac{z^2}{\sigma_z^2} } } \\ 
\Rightarrow f(x,y,z) &= \brak{\frac{e^{-x^{2}/2{\sigma_x}^{2}}}{\sigma_x\sqrt{2\pi}}} \brak{\frac{e^{-y^{2}/2{\sigma_y}^{2}}}{\sigma_y\sqrt{2\pi}}} \brak{\frac{e^{-z^{2}/2{\sigma_z}^{2}}}{\sigma_z\sqrt{2\pi}}} \\
\Rightarrow f(x,y,z) &= f_x(x) f_y(y) f_z(z)
\end{align}
$\therefore x,y,z$ are independent.
\end{document}